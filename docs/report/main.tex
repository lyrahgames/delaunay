\documentclass[10pt, twoside, fleqn]{article}

\usepackage[utf8]{inputenc}
\usepackage[T1]{fontenc}
\usepackage{microtype}
\usepackage{times}
\linespread{1.15}

\usepackage[a4paper, left=45mm, right=35mm, top=40mm, bottom=35mm]{geometry}

\usepackage[english]{babel}

\usepackage{emptypage}
\usepackage{multicol}
\usepackage{fancyhdr}
\usepackage{lastpage}
\fancypagestyle{sectionstyle}{
  \fancyhf{}
  % \fancyfoot[C]{\footnotesize\bigskip\thepage/\pageref{LastPage}}
  % \fancyfoot{}
  \fancyfoot[C]{\footnotesize\bigskip\thepage}
  \renewcommand{\footrulewidth}{0.5pt}
  \renewcommand{\headrulewidth}{0pt}
}
\fancypagestyle{mainstyle}{
  \fancyhf{}
  \fancyfoot[C]{\footnotesize\bigskip\thepage}
  % \fancyhead[LO,RE]{\footnotesize \thetitle} %left
  % \fancyhead[RO,LE]{\footnotesize \theauthor} %right
  % \fancyhead[LO,RE]{\footnotesize \ \smallskip}
  \fancyhead[RO]{\footnotesize \rightmark \smallskip} %right
  \fancyhead[LE]{\footnotesize \leftmark \smallskip} %right
  \renewcommand{\headrulewidth}{0.5pt}
  \renewcommand{\footrulewidth}{0.5pt}
}
\pagestyle{mainstyle}

% Figures, Graphics, and Colors
\usepackage{graphicx}
\usepackage{float}
\usepackage[hang,footnotesize]{caption}
\usepackage[font=footnotesize]{subcaption}
\usepackage[dvipsnames]{xcolor}

% Mathematics
\usepackage{amsmath}
\usepackage{amssymb}
\usepackage{amsfonts}
\usepackage{mathtools}
\allowdisplaybreaks
\usepackage{dsfont}
\usepackage{euscript}
\usepackage{exscale}

\usepackage{enumitem}
\usepackage{listings}
\usepackage{tcolorbox}

% References, Quotes, Labels
\usepackage[autostyle,german=guillemets,english=american]{csquotes}
\usepackage[backend=bibtex,style=authoryear-comp,dateabbrev=false]{biblatex}
\usepackage{tocbibind}
\usepackage{url}
\urlstyle{same}
\usepackage[pdfusetitle]{hyperref}
\hypersetup{
  colorlinks,
  linkcolor={blue!70!black},
  citecolor={blue!70!black},
  urlcolor={blue!70!black}
}

% Custom
\let\oldsection\section
\renewcommand*\section{%
  \cleardoublepage
  \thispagestyle{sectionstyle}\oldsection}
\let\oldpagenumbering\pagenumbering
\renewcommand*\pagenumbering[1]{%
  \cleardoublepage
  \oldpagenumbering{#1}}

% Content
\title{Delaunay Triangulation}
\author{Markus Pawellek}
\date{\today}
\bibliography{references}
\hypersetup{
  % pdftex,
  % pdfauthor={Markus Pawellek},
  % pdftitle={The Title},
  pdfsubject={Delaunay Triangulation},
  pdfkeywords={delaunay, triangulation, tessellation, subdivision, voronoi, n-dimensional},
  % pdfproducer={Latex with hyperref},
  % pdfcreator={pdflatex, or other tool}
}

\newcommand{\naturalnumbers}{\mathds{N}}
\newcommand{\realnumbers}{\mathds{R}}
\newcommand{\norm}[1]{\left\| #1 \right\|}
\newcommand{\scalarproduct}[2]{\left\langle #1 \middle| #2 \right\rangle}

\begin{document}
  \maketitle
  \thispagestyle{empty}

  \bigskip
  \hrule
  \medskip
  \begin{abstract}
    Lorem ipsum dolor sit amet, consectetur adipisicing elit, sed do eiusmod
    tempor incididunt ut labore et dolore magna aliqua. Ut enim ad minim veniam,
    quis nostrud exercitation ullamco laboris nisi ut aliquip ex ea commodo
    consequat. Duis aute irure dolor in reprehenderit in voluptate velit esse
    cillum dolore eu fugiat nulla pariatur. Excepteur sint occaecat cupidatat non
    proident, sunt in culpa qui officia deserunt mollit anim id est laborum.
  \end{abstract}
  \medskip
  \hrule
  \bigskip

  \pagenumbering{roman}
  \tableofcontents

  \pagenumbering{arabic}
  \section{Introduction}
  \section{Background}
    \subsection{Graph Theory}
    \subsection{Geometry}
      \subsubsection{Circumcircle}

        \begin{align*}
          \norm{a - m} &= r \\
          \norm{b - m} &= r \\
          \norm{c - m} &= r
        \end{align*}

        \begin{align*}
          \norm{a - m}^2 &= r^2 \\
          \norm{b - m}^2 &= r^2 \\
          \norm{c - m}^2 &= r^2
        \end{align*}

        \begin{align*}
          \norm{\hat{m}}^2 &= r^2 \\
          \norm{\hat{b} - \hat{m}} &= r^2 \\
          \norm{\hat{c} - \hat{m}} &= r^2
        \end{align*}

        \begin{align*}
          \norm{\hat{b}}^2 + \norm{\hat{m}}^2 - 2 \scalarproduct{\hat{b}}{\hat{m}} &= r^2 \\
          \norm{\hat{c}}^2 + \norm{\hat{m}}^2 - 2 \scalarproduct{\hat{c}}{\hat{m}} &= r^2
        \end{align*}

        \begin{align*}
          \norm{\hat{b}}^2 - 2 \scalarproduct{\hat{b}}{\hat{m}} &= 0 \\
          \norm{\hat{c}}^2 - 2 \scalarproduct{\hat{c}}{\hat{m}} &= 0
        \end{align*}

        \begin{align*}
          \norm{\hat{b}}^2 - 2 \scalarproduct{\hat{b}}{\hat{m}} &= 0 \\
          \norm{\hat{c}}^2 - 2 \scalarproduct{\hat{c}}{\hat{m}} &= 0
        \end{align*}

        \begin{align*}
          \scalarproduct{\hat{b}}{\hat{m}} &= \frac{1}{2}\norm{\hat{b}}^2 \\
          \scalarproduct{\hat{c}}{\hat{m}} &= \frac{1}{2}\norm{\hat{c}}^2
        \end{align*}

        \[
          \begin{pmatrix}
            \hat{b} & \hat{c}
          \end{pmatrix}
          ^\mathrm{T}
          \hat{m}
          = \frac{1}{2}
          \begin{pmatrix}
            \norm{\hat{b}}^2 \\
            \norm{\hat{c}}^2
          \end{pmatrix}
        \]

        \[
          \hat{m} =
          \frac{1}{2\det\begin{pmatrix}\hat{b} & \hat{c}\end{pmatrix}}
          \mathrm{adj} \begin{pmatrix}\hat{b} & \hat{c}\end{pmatrix}^\mathrm{T}
          \begin{pmatrix}
            \norm{\hat{b}}^2 \\
            \norm{\hat{c}}^2
          \end{pmatrix}
        \]

        \[
          \hat{m} =
          \frac{1}{2 (u_x v_y - u_y v_x)}
          \begin{pmatrix}
            v_y (u_x^2 + u_y^2) - v_x (v_x^2 + v_y^2) \\
            u_x (v_x^2 + v_y^2) - u_y (u_x^2 + u_y^2)
          \end{pmatrix}
        \]

      \subsubsection{Circumsphere} % (fold)
        \begin{align*}
          \norm{a - m} &= r \\
          \norm{b - m} &= r \\
          \norm{c - m} &= r \\
          \norm{d - m} &= r
        \end{align*}

        \begin{align*}
          \norm{a - m}^2 &= r^2 \\
          \norm{b - m}^2 &= r^2 \\
          \norm{c - m}^2 &= r^2 \\
          \norm{d - m}^2 &= r^2
        \end{align*}

        \begin{align*}
          \norm{\hat{m}}^2 &= r^2 \\
          \norm{\hat{b} - \hat{m}} &= r^2 \\
          \norm{\hat{c} - \hat{m}} &= r^2 \\
          \norm{\hat{d} - \hat{m}} &= r^2
        \end{align*}

        \begin{align*}
          \norm{\hat{b}}^2 + \norm{\hat{m}}^2 - 2 \scalarproduct{\hat{b}}{\hat{m}} &= r^2 \\
          \norm{\hat{c}}^2 + \norm{\hat{m}}^2 - 2 \scalarproduct{\hat{c}}{\hat{m}} &= r^2 \\
          \norm{\hat{d}}^2 + \norm{\hat{m}}^2 - 2 \scalarproduct{\hat{c}}{\hat{m}} &= r^2
        \end{align*}

        \begin{align*}
          \norm{\hat{b}}^2 - 2 \scalarproduct{\hat{b}}{\hat{m}} &= 0 \\
          \norm{\hat{c}}^2 - 2 \scalarproduct{\hat{c}}{\hat{m}} &= 0 \\
          \norm{\hat{d}}^2 - 2 \scalarproduct{\hat{c}}{\hat{m}} &= 0
        \end{align*}

        \begin{align*}
          \norm{\hat{b}}^2 - 2 \scalarproduct{\hat{b}}{\hat{m}} &= 0 \\
          \norm{\hat{c}}^2 - 2 \scalarproduct{\hat{c}}{\hat{m}} &= 0 \\
          \norm{\hat{d}}^2 - 2 \scalarproduct{\hat{d}}{\hat{m}} &= 0
        \end{align*}

        \begin{align*}
          \scalarproduct{\hat{b}}{\hat{m}} &= \frac{1}{2}\norm{\hat{b}}^2 \\
          \scalarproduct{\hat{c}}{\hat{m}} &= \frac{1}{2}\norm{\hat{c}}^2 \\
          \scalarproduct{\hat{d}}{\hat{m}} &= \frac{1}{2}\norm{\hat{d}}^2
        \end{align*}

        \[
          \begin{pmatrix}
            \hat{b} & \hat{c} & \hat{d}
          \end{pmatrix}
          ^\mathrm{T}
          \hat{m}
          = \frac{1}{2}
          \begin{pmatrix}
            \norm{\hat{b}}^2 \\
            \norm{\hat{c}}^2 \\
            \norm{\hat{d}}^2
          \end{pmatrix}
        \]

        \[
          \hat{m} =
          \frac{1}{2\det\begin{pmatrix}\hat{b} & \hat{c} & \hat{d}\end{pmatrix}}
          \mathrm{adj} \begin{pmatrix}\hat{b} & \hat{c} & \hat{d}\end{pmatrix}^\mathrm{T}
          \begin{pmatrix}
            \norm{\hat{b}}^2 \\
            \norm{\hat{c}}^2 \\
            \norm{\hat{d}}^2
          \end{pmatrix}
        \]

      \subsubsection{Circumscribed $n$-Sphere}
        \[
          \norm{x_0 - m} = r
        \]
        \[
          \norm{x_i - m} = r
        \]
        \[
          \norm{x_i - m}^2 = r^2
        \]
        \[
          \norm{\hat{x}_i - \hat{m}}^2 = r^2
        \]
        \[
          \norm{\hat{x}_i}^2 + \norm{\hat{m}}^2 - 2 \scalarproduct{\hat{x}_i}{\hat{m}} = r^2
        \]
        \[
          \norm{\hat{x}_i}^2 - 2 \scalarproduct{\hat{x}_i}{\hat{m}} = 0
        \]
        \[
          \scalarproduct{\hat{x}_i}{\hat{m}} = \frac{1}{2}\norm{\hat{x}_i}^2
        \]
        \[
          \begin{pmatrix}
            \hat{x}_1 & \ldots & \hat{x}_n
          \end{pmatrix}
          ^\mathrm{T}
          \hat{m}
          = \frac{1}{2}
          \begin{pmatrix}
            \norm{\hat{x}_1}^2 \\
            \vdots \\
            \norm{\hat{x}_n}^2
          \end{pmatrix}
        \]
        Here, we should not try to build the adjoint.
        For higher dimensions, it is much more efficient to use a simple matrix solver, like $LU$ decomposition or Cholesky.

    \subsection{Delaunay Triangulation and Tessellation}
      Triangulation, Tessellation, Simplicialization, Subdivision, Mesh Generation
      Two and Three Dimensions
  \section{Algorithms and Data Structures}
    \subsection{Hash Map}
    \subsection{Triangle Mesh}
    \subsection{Quad-Edge Data Structure}
    \subsection{Radix Sort}
    \subsection{Linear Morton Sort}
    \subsection{Linear Floating-Point Quad-Tree}
    \subsection{Bowyer-Watson Algorithm}
    \subsection{Guibas-Stolfi Incremental Algorithm}
    \subsection{Guibas-Stolfi Divide-and-Conquer Algorithm}
    \subsection{Dwyer Algorithm}
  \section{Design and Implementation}
    \subsection{API}
    \subsection{Robustness}
    \subsection{Incremental Algorithm}
    \subsection{Divide-and-Conquer Algorithm}
    \subsection{Multidimensional Tessellation}
  \section{Tests and Testscenes}
    \subsection{Uniform Rectangular Distribution}
    \subsection{Gaussian Distribution}
    \subsection{Robustness Tests}
  \section{Benchmarks}
  \section{Examples}
    \subsection{Image Mosaic and Tessellation}
    \subsection{Fluids for Ray Tracing}
    \subsection{Finite Element Method}
    \subsection{Pareto Frontiers}
  \section{Results}
  \section{Conclusions}

  \nocite{*}
  \printbibliography[heading=bibintoc]
\end{document}